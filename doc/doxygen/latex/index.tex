Controla una \hyperlink{class_nau}{Nau} a l\textquotesingle{}espai evitant i atacant Meteorits i Naus Enemigues \begin{DoxyAuthor}{Author}
Miquel Farreras Casamort
\end{DoxyAuthor}
El \hyperlink{class_joc}{Joc} conté quatre tipus d\textquotesingle{}\hyperlink{interface_objecte_joc}{Objecte\+Joc}\+:
\begin{DoxyItemize}
\item \hyperlink{class_nau}{Nau}\+: \hyperlink{class_nau}{Nau} pròpia controlada per l\textquotesingle{}usuari.
\item \hyperlink{class_nau_enemiga}{Nau\+Enemiga}\+: \hyperlink{class_nau}{Nau} controlada per la màquina (A\+I).
\item \hyperlink{class_meteorit}{Meteorit}\+: Objectes controlats per la màquina, que van a la deriva.
\item \hyperlink{class_raig_laser}{Raig\+Laser}\+: Raig disparat per \hyperlink{class_nau}{Nau} i \hyperlink{class_nau_enemiga}{Nau\+Enemiga}, que destrueix Meteorits i Naus.
\end{DoxyItemize}

\subsection*{Funcionament general del \hyperlink{class_joc}{Joc}\+: }

\hyperlink{class_nau}{Nau}\+:
\begin{DoxyItemize}
\item Rota sobre si mateixa. Té un coet propulsor que l\textquotesingle{}impulsa endavant.
\item Quan abandoni l\textquotesingle{}espai visible per l\textquotesingle{}usuari, per algun dels costats de la finestra, apareixerà pel costat invers, conservant el moviment que portava.
\item Té forma de triangle isòsceles.
\item Si s\textquotesingle{}acaben les vides de la \hyperlink{class_nau}{Nau}, la partida s\textquotesingle{}ha acabat.
\item Un cop s\textquotesingle{}ha començat a moure en una direcció, es continua movent en aquesta direcció durant un temps determinat mentre l\textquotesingle{}usuari no intervingui, simulant la ingravidesa de l\textquotesingle{}espai, i al mateix temps, facilitant el control de la nau.
\end{DoxyItemize}

\hyperlink{class_nau_enemiga}{Nau\+Enemiga}\+:
\begin{DoxyItemize}
\item Hi ha una \hyperlink{class_nau_enemiga}{Nau\+Enemiga} que intenta, per qualsevol mitjà (disparant i col·lisionant), destruir la nau de l\textquotesingle{}usuari.
\item Té el mateix comportament que la \hyperlink{class_nau}{Nau} espacial controlada per l\textquotesingle{}usuari.
\item Excepte\+:
\begin{DoxyItemize}
\item L\textquotesingle{}objectiu d\textquotesingle{}aquesta nau és destruir la nau de l\textquotesingle{}usuari.
\item Per a destruir la nau de l\textquotesingle{}usuari dispara rajos làser i la persegueix.
\end{DoxyItemize}
\item La \hyperlink{class_nau}{Nau} i la \hyperlink{class_nau_enemiga}{Nau\+Enemiga} poden disparar Rajos\+Laser.
\end{DoxyItemize}

\hyperlink{class_raig_laser}{Raig\+Laser}\+:
\begin{DoxyItemize}
\item Els \hyperlink{class_raig_laser}{Raig\+Laser} poden col·lisionar amb Meteorits o amb les Naus.
\end{DoxyItemize}

\hyperlink{class_meteorit}{Meteorit}\+:
\begin{DoxyItemize}
\item És un objecte que es mou amb una velocitat i direcció pseudoaleatòries (entre un rang determinat)
\item Té una forma d\textquotesingle{}un polígon irregular, pseudoaleatòria
\item Els Meteorits no col·lisionen entre si, s\textquotesingle{}atravessen.
\item Quan un \hyperlink{class_meteorit}{Meteorit} gran col·lisiona amb un \hyperlink{class_raig_laser}{Raig\+Laser} o una \hyperlink{class_nau}{Nau}, es divideix en dos Meteorits petits.
\item Si xoca contra una nau, destrueix la nau amb la qual ha xocat.
\item Quan un \hyperlink{class_meteorit}{Meteorit} petit col·lisiona amb un \hyperlink{class_raig_laser}{Raig\+Laser} o una \hyperlink{class_nau}{Nau}, aquest desapareix.
\end{DoxyItemize}

\subsection*{Descripció general\+: }

Inicialment, hi ha diversos \hyperlink{class_meteorit}{Meteorit} grans dispersats per tot l\textquotesingle{}espai i la \hyperlink{class_nau_enemiga}{Nau\+Enemiga} a prop d\textquotesingle{}algun extrem de l\textquotesingle{}espai. La \hyperlink{class_nau}{Nau} controlada per l\textquotesingle{}usuari està al centre. Hi ha un nombre màxim de \hyperlink{class_meteorit}{Meteorit} que poden estar dins l\textquotesingle{}espai del joc al mateix temps. Quan es comença, hi ha un número determinat de meteorits grans, que es va augmentant fins al límit. A partir de llavors, cada vegada que es destrueixi un meteorit n\textquotesingle{}apareix un de nou.

Quan la \hyperlink{class_nau}{Nau} es destrueix i encara té vides, llavors tarda 5 segons a reaparèixer. Només hi ha una \hyperlink{class_nau_enemiga}{Nau\+Enemiga} a l\textquotesingle{}espai. Cada vegada que sigui destruïda, despres de 5 segons n\textquotesingle{}apareix una de nova.

El joc té un sistema de puntuació i de vides. L\textquotesingle{}usuari sempre comença amb 3 vides i 0 punts. A mesura que va destruint meteorits i naus enemigues, augmenta la seva puntuació, segons aquestes ponderacions\+: meteorit gran 50 punts, meteorit petit 20 punts, nau enemiga 100 punts.

Quan s\textquotesingle{}acaben les vides de la \hyperlink{class_nau}{Nau}, es mostra \char`\"{}\+Game Over\char`\"{} juntament amb la puntuació.

\subsection*{Controls\+: }


\begin{DoxyItemize}
\item W\+: impulsar cap endavant
\item A\+: rotar cap a l\textquotesingle{}esquerra
\item D\+: rotar cap a la dreta
\item Espai\+: disparar rajos làser
\item E\+S\+C\+: sortir del joc 
\end{DoxyItemize}